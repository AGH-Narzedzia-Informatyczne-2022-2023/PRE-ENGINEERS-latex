\section{Zuzanna Konopka}
\label{sec:zuza}
\begin{flushleft}
Jako pasjonatka kotów nie mogę nie powiedzieć \\
kilku słów o moich ulubieńcach. (see Figure~\ref{fig:kotek}).

\begin{figure}[htbp] 
    \centering
    \includegraphics[width=0.5\textwidth]{pictures/kotek.jpg} 
    
    \caption{Rozkosz wylewa się poza obrazek}
    \label{fig:kotek}
\end{figure}

\textbf{{\Huge Ciekawostki o kotkach:}}

\vspace {0.60cm}
    
    Wiele osób uważa, że jak ma kota, to wie o nim wszystko.
    \textbf{OTÓŻ NIE TYM RAZEM}. Koty to niesamowite zwierzęta. Ich mruczenie potrafi zmiękczyć serce niejednego twardziela.Koty wykorzystują swoje wąsy by sprawdzić, czy przejście jest na tyle szerokie, że się w nim zmieszczą. Nie ma dwóch identycznych kocich nosów - każdy z nich jest tak unikalny, jak u ludzi linie papilarne i daje inny odcisk.
  
\vspace {0.50cm}
{{\huge A wiesz jak poprawić kotu nastrój?}}
\vspace {0.50cm}

    Zdarzają się dni, podczas których widzisz, że Twój kot ewidentnie nie ma dobrego nastroju. Sprzyjają temu również ciemne, jesienne tygodnie, które swoją deszczową aurą sprawiają, że pupil najchętniej przespałby niemal całą dobę.
    
\vspace {0.50cm}
    Warto na co dzień dbać o harmonię w życiu kota i tak organizować jego przestrzeń życiową, aby nie czuł się sfrustrowany i znudzony. Przede wszystkim należy zacząć od poznania potrzeb typowych dla gatunku. Jako zwierzę terytorialne kot bardzo lubi patrolować swoją okolicę. W tym celu w domu należy znaleźć miejsce na stworzenie ciekawych kryjówek, podwyższonych miejsc do obserwacji i różnego rodzaju wysokich półeczek wygodnych do przemieszczania się powyżej poziomu podłogi. Do korzystania z tego typu asortymentu zazwyczaj nie trzeba przekonywać tych czworonogów, ale jeśli widzisz, że Twój kot jest ostrożny, daj mu trochę czasu na zapoznanie się z przedmiotami.\underline {Możesz również pokropić elementy wyciągiem z kocimiętki.}
 
\newpage
\vspace {0.80cm}

\begin{table}
\caption{Rasy kotów}
\label{tab:Rasy_kotów}
\begin{center}
\begin{tabular}{|c||c|c|c|}
\hline Rasa & Maść & Charakter \\ \hline \hline
KHAO MANEE & biało-pręgaty & spokojny \\ \hline
KORAT & srebrzystoniebieski & zrównoważony \\ \hline
KOT BIRMAŃSKI & biały & aktywny  \\ \hline
\end{tabular}
\end{center}
\end{table}



\noindent {{\Large Coś czego na pewno nie wiedziałeś o kotach:}}

\begin{itemize}
  \item Koty nie czują słodkiego smaku,
  \item Koty mogą obracać uszami o 180 stopni,
  \item Są w stanie wyczuć trzęsienie ziemi na długo przed tym, gdy poczują je ludzie
  \item Koty mają trzy powieki!!!
\end{itemize}

\maketitle 
    Wbrew pozorom, koty świetnie się uczą. Wymagają tylko większej precyzji od swoich opiekunów. Warto nauczyć kota i kilku przydatnych czynności i wychować kota, tak, aby żyło nam się z nim lepiej. Koty można nauczyć rozpoznawania swojego imienia, jak również przychodzenia na komendę.

\noindent Proste sztuczki, których możesz nauczyć swojego kota:

\begin{enumerate}
  \item Siad
  \item Aport
   \item Reakcja na swoje imię
    \item Akceptacji innych zwierząt w domu
\end{enumerate}


Przeciętny kot przesypia 2/3 swojego życia. Tą 1/3 życia pragnie przespać student, ale to nawet mu nie wychodzi.

Nawet nie wiecie jak bardzo student chciałby być kotem. Koty są leniwe, ale tak inteligentne, że potrafiłyby rozwiązać takie trudne równanie:  


$$\int_{c}^{d} \left[ \int_{u(y)}^{v(y)} f(x,y)dx \right] dy$$

Uwierzcie mi lub nie, ale student tak szybko tego nie rozwiąże jak kot. 
Następnym razem na kolokwium pójdzie za mnie kot. Wtedy może i zdam...

\end{flushleft}