\section{Martyna Mulawa}
\label{sec:martynamulawa}



\vspace {0.51cm}
\begin{flushleft} Gdzie jest słonko kiedy śpi? Czy wilk zawsze bywa zły? Dokąd tupta nocą jeż? Możesz wiedzieć jeśli chcesz!\end{flushleft}


\begin{flushleft}  Jeż zachodni, jeż europejski – gatunek ssaka łożyskowego z rodziny jeżowatych. Występuje w klimacie umiarkowanym na terenie od zachodniej Europy po zachodnią Polskę, Skandynawię i północno-zachodni obszar europejskiej części Rosji.\end{flushleft}

\begin{figure}[htbp]
    \centering
    \includegraphics[width=0.7\textwidth]{pictures/jezyk.jpg}
    \caption{Jeżyk!}
    \label{fig:jezyk}
\end{figure}

\newpage
\noindent Co ma każdy jeżyk:
\begin{itemize}
  \item kolce
  \item 4  łapki
  \item uszka 
  \item nosek
\end{itemize}

\noindent Gatunki jeży:
\begin{enumerate}
  \item zachodni
  \item wschodni
  \item anatolijski
\end{enumerate}

\begin{table}[htbp]
\centering
\begin{tabular}{||c c c c||} 
 \hline
 1 & 2 & 3 & 4 \\ [0.5ex] 
 \hline\hline
 1 & Góra lewo & Góra środek & Góra prawo \\ 
 \hline
 2 & Środek lewo & Środek środek & Środek prawo \\
 \hline
 3 & Dół lewo & Dół środek & Dół prawo \\
 \hline


 \hline
\end{tabular}
\label{tab:12}
\caption{Tabelka z nazwaniem komórek}
\end{table}


\noindent Wzór skróconego mnożenia
$ (a+b)^2=a^2+2ab+b^2$

\noindent Złota liczba
$\frac{1+\sqrt5}{2}$



