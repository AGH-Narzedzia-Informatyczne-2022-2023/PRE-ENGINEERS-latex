\section{Anna Jochymczyk}
\label{sec:ania}

Tekst powstał dzięki współpracy z pewnym mądrym psem (see Figure~\ref{fig:piesek}).

\begin{figure}[htbp] 
    \centering
    \includegraphics[width=0.2\textwidth]{pictures/piesek.jpg} 
    
    \caption{Pełne skupienie}
    \label{fig:piesek}
\end{figure}

\textbf{{\large Czy mój pies umie czytać?}}:

\vspace {0.51cm}
\maketitle
    

    Dla większośći ludzi odpowiedź wydaje się być oczywista. Ale czy na pewno? Skoro ludzie widząc lub słysząc słowa automatyczenie przypisują im znaczenie, dlaczego psy pod tym względem miałyby być gorsze ? Wykonywanie komend działa przecież na podobnej zasadzie, psy zapamiętują gesty lub zasłyszane słowa, reagując na nie w nauczony przez nas sposób. Przeprowadziłam pewien eksperyment. Na zwykłej kartce papieru napisałam dobrze znaną mojemu psu komendę "siad". Pokazując mu kartkę powtarzałam słowo, po kilku sesjach pies widząc napis na kartce wykonywał polecenie bez jakiejkolwiek podpowiedzi. Ten sam proces powtórzyłam z innymi komendami (napisy były czytelne i wyraźnie się różniły). Po kilku tygodniach pies rozróżniał kilkanaście napisów i widząc je wykonywał polecenia.     Pokazuje to to, że zwierzęta mogą zapamiętywać dane obrazy, a nawet słowa w różnych językach, są w stanie rozróżniać długość zdań, a nawet pojedynczych słów i po nauce przypisywać im znaczenia. 
\vspace {0.51cm}

\maketitle    Psy są w stanie przypisać podstawowe znaczenie danemu słowu. Odpowiadając na tytułowe pytanie możliwym jest nauczenie psa czytania prostych zwrotów, a raczej nauczenie go znaczenia napisanych wyrazów.  Metodami naprowadzania, kształtowania i wyłapywania  uczy się pożądanej przez nas reakcji na kartkę z napisanym na niej słowem. Przy użyciu klikera da się naprowadzić psa na znaczenie innych słów, a nawet całych zdań. Niestety szybkość z jaką się uczą nie pozwala na przyswojenie wszystkiego w ciągu ich życia.




\begin{table}[htbp]
\centering
\begin{tabular}{||c c c c||} 
 \hline
 1 & 2 & 3 & 4 \\ [0.5ex] 
 \hline\hline
 1 & Nie & mam & pomysłu \\ 
 \hline
 2 & co & wpisać & w  \\
 \hline
 3 & tą & tabelę & , \\
 \hline
 4 & więc & wpisuję & słowa \\
 \hline
 5 & i  & znaki & . \\
 \hline
\end{tabular}
\label{tab:random_numbers2}
\caption{Pod tą tabelką jest podpis}
\end{table}


\noindent Poprzez jakie metody psy najłatwiej się u uczą?

\begin{itemize}
  \item kształtowanie
  \item wyłapywanie
  \item naprowadzanie
\end{itemize}

\noindent Czynniki wpływające na czas nauki?


\begin{enumerate}
  \item złożoność zadania
  \item wcześniejsze doświadczenia
   \item umiejętność radzenia sobie z presją ( rozumianą jako oczekiwanie poprawnego wykonania polecenia)
    \item wrażliwość emocjonalna
    \item zdolność łączenia sytuacji i wyciągania wniosków
\end{enumerate}


Nad rozwiązaniem tego wyrażenia nadal pracuje mój pies:


 $\lim_{n \to \infty}
\sum_{k=3}^n \frac{1}{k^9}
= \frac{\pi^2}{2}$